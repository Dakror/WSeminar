\documentclass[12pt]{article}
\usepackage[ngerman]{babel}
\usepackage{amsmath}
\usepackage[utf8]{inputenc}
\usepackage{hyperref}
\usepackage{setspace}
\setstretch{1.5}
\usepackage[paper=a4paper,left=25mm,right=35mm,top=25mm,bottom=2mm]{geometry}

\hypersetup{
    colorlinks,
    citecolor=black,
    filecolor=black,
    linkcolor=black,
    urlcolor=black
}

\begin{document}

\title{Extremwertproblem Wegfindung \\  \large{\enspace Vergleich verschiedener Algorithmen}}

\author{Maximilian Stark}

\maketitle
\thispagestyle{empty}
\clearpage

\tableofcontents
\clearpage

\section{Einleitung}
\newpage

\section{Grundlagen und Terminologie}
Zu Beginn werden in diesem Abschnitt die grundlegenden Begriffe der Graphen-Theorie geklärt. Auch Fachbegriffe aus der Implementierung durch die Informatik werden erläutert.
\\
Das Fundament der Graphen-Theorie ist der namensgebende \textit{Graph} $G\; = \{\;V,\;E\;\}$, welcher aus einer Menge von \textit{Knoten} $V$ (von engl. "`Vertex") und aus einer Menge \textit{Kanten} $E$ (von engl. "`Edge").\\
  Zeichnerisch werden \textit{Knoten} als Punkte oder Kreise dargestellt; \textit{Kanten} als Verbindungslinien zwischen zwei \textit{Knoten}. Jede \textit{Kante} hat einen \textit{Startknoten} und einen \textit{Endknoten}. Wenn von einer \textit{gerichteten Kante} die Rede ist, lässt sich das als Pfeil interpretieren, da die Verbindung monodirektional gilt. Ebenso gibt es die \textit{gewichteten Kanten}, denen nicht nur zwei \textit{Knoten} zugeordnet werden, sondern zusätzlich noch ein Gewicht $w$ (von engl. "`Weight"), ein Zahlenwert, der als Kosten der Beziehung zwischen den beiden \textit{Knoten} gesehen werden kann.
\\
In der Wegfindung ist ein \textit{Weg} $P$ (von engl. "`Path") als geordnete Abfolge von \textit{Knoten} definiert. Da in der Regel jedes \textit{Knoten}-Paar nur einfach verbunden ist, reicht in der Implementierung dieser Ansatz aus.


\section{Aufbau und Bedienung des Programms}
\newpage
-
\newpage

\section{Konstruktion des Graphen}
\newpage

\section{Visuelles Layout des Graphen}
\newpage

\section{Wegfindungs-Algorithmen}
\newpage

\subsection{Gröbste Züge von Intelligenz: Tiefensuche}
\newpage
-
\newpage

\subsection{Heuristik als Mittel zum Ziel: Der Dijkstra-Algorithmus}
\newpage
-
\newpage

\subsection{Der Allstar: Der A*-Algorithmus}
\newpage
-
\newpage

\section{Vergleichsstatistik und Fazit}
\newpage
-
\newpage

\section{Schluss}

\end{document}
